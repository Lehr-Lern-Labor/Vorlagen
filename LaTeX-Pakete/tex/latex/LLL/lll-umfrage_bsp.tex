\documentclass[layout=lll]{lll-layout}
% Optionen: printlsg (Lösungen nach Aufgabe), lsgseite (Lösungsseite am Ende)
\usepackage{lll-umfrage}

\titel{Umfrage}
\datum{28.08.2020}

\begin{document}
\renewcommand{\matrixWidth}{0.55\textwidth}
Alle deine Angaben werden anonymisiert. Wir werden also später nicht mehr feststellen können, wer welche Angaben gemacht hat.

Schon jetzt vielen Dank für deine Teilnahme!

\rule{\textwidth}{1pt}

\begin{Form}
\frage{Frage für Formulare}
\einfach{Einfache Antwort}
\freitext{Freitext}
\auswahl{Auswahl}{Option 1, Option 2, Option 3}
\einfachauswahl{Einfachauswahl}{Option 1, Option 2, Option 3}
\mehrfachauswahl{Mehrfachauswahl}{Option 1, Option 2, Option 3}

\einfach*{Einfache Antwort (ganzzeilig)}
\freitext*{Freitext (ganzzeilig)}
\auswahl*{Auswahl (ganzzeilig)}{Option 1, Option 2, Option 3}
\einfachauswahl*[name=Einfachauswahl (ganzzeilig)]{Einfachauswahl\hfill(ganzzeilig)}{Option 1, Option 2, Option 3}
\mehrfachauswahl*{Mehrfachauswahl (ganzzeilig)}{Option 1, Option 2, Option 3}

\begin{einfachliste}{Singlechoice - einfachliste}
	\option[Option 1]{Option 1\hfill *}
	\option{Option 2}
\end{einfachliste}

\begin{einfachliste*}{2}{Singlechoice - einfachliste (ganzzeilig)}
	\option{Option 1\hfill *}
	\option{Option 2}
	\option{Option 3}
	\option{Option 4}
	\option{Option 5}
\end{einfachliste*}

\begin{mehrfachliste}{Multiplechoice - mehrfachliste}
	\option[Option 1]{Option 1\hfill *}
	\option{Option 2}
\end{mehrfachliste}

\begin{mehrfachliste*}{2}{Multiplechoice - mehrfachliste (ganzzeilig)}
	\option[Option 1]{Option 1\hfill *}
	\option{Option 2}
	\option{Option 3}
	\option{Option 4}
	\option{Option 5}
\end{mehrfachliste*}

\begin{matrixfrage}{5}{Matrixfrage}
	\head{1}\head{2}\head{3}\head{4}\head{5}\\
	\option[Option 1]{Option 1\hfill *}
	\option{Option 2}
	\option{Option 3}
\end{matrixfrage}


\end{Form}





%\begin{Form}[action=mailto:?subject=Umfrage&body=Vielen Dank für Ihre Teilnahme! Bitte senden Sie das Formular an lehr-lern-labor@informatik.kit.edu.]
%\begin{einfachsauwahl}{Warum ist die Banane krumm?}%
%\begin{multicols}{3}
%	\choice{Pinselstift}
%	\choice{Bleistift}
%	\choice{Radiergummi}
%\end{multicols}
%\end{einfachsauwahl}
%
%\begin{mehrfachliste}[noitemsep]{Warum ist die Erde keine Scheibe?}
%	\choice{Pinselstift}
%	\choice{Bleistift}
%	\choice{Radiergummi}
%\end{mehrfachliste}
%
%
%%Vorname \TextField[bordercolor=black,multiline]{ }
%%
%%Vorname \TextField[bordercolor=black,name=sbvdlj]{ }
%%
%%\TextField[bordercolor=black]{Nachname}
%%
%%\TextField[bordercolor=black,multiline]{Nachname}
%%
%%\ChoiceMenu[combo]{Anrede}{Frau, Herr}
%
%\freitext{Das wollte ich loswerden:}
%
%\auswahl{Daher kenn ich euch:}{BNN, Schule, Freunde, Familie}
%
%\einfach{Wie heißt du?}
%
%
%%\checkchoice{Erste Veranstaltung am KIT}
%%
%%\choosequestion{Wie alt bist du?}
%%\choosechoice{0-100} \choosechoice{100-200} \choosechoice{200-300}
%
%
%\blindtext
%
%\renewcommand{\matrixWidth}{0.5\textwidth}
%
%
%\begin{matrixfrage}{3}
%%\foreach \n in {#4}{&\multicolumn{1}{c}{\textbf{\n}}}\\
%Welche Frage passt? \head{Top} \head{Naja}  \head{Mies}\\
%\frage{Geht das so?}
%\frage{Geht das auch so?}
%\frage{Geht das so?}
%\end{matrixfrage}
%
%\begin{matrixfrage}{4}
%%\foreach \n in {#4}{&\multicolumn{1}{c}{\textbf{\n}}}\\
%Welche Frage passt dieses Mal??\head{I} \head{II} \head{keine\par Angabe}  \head{III}\\
%\frage{Geht das so?}
%\frage{Geht das auch so?}
%\frage{Geht das so?}
%\end{matrixfrage}
%
%\Submit[export=pdf]{Absenden}
%\end{Form}


















%\kasten[symb=\faInfo]{Merkkästen für wichtige Infos}{\blindtext}
%
%
%\kombi[symb=\faLightbulbO]{Hier findet man eine Übersicht über die verfügbaren Makros, die auch direkt Lösungen enthalten.}{
%	Lückentext:
%	
%	Ich bin eine \luecke{Lücke}.	
%	
%	Kariert:
%	
%	\kariert[2]{Eine Rechnung mit einer optionalen Kästchen-Anzahl (Höhe)}
%	
%	Liniert:
%	
%	\liniert[2]{eine bestimmte Anzahl leerer Linien für Freitext-Antworten}
%	
%	Blanko:
%	
%	\blanko[2]{freier Platz für individuelle Antworten (Angbabe in Zeilen)}
%}
%\kombi{auch mehrspaltiger Satz ist möglich:}{
%	\begin{spalten}
%		\kariert[3]{linke Spalte}
%		\liniert[2]{rechte Spalte}
%	\end{spalten}
%	\begin{spalten}[3]
%		\kariert[3]{linke Spalte}
%		\blanko[3]{mittlere Spalte}
%		\liniert[2]{rechte Spalte}
%	\end{spalten}
%}
%
%\newpage
%\kombi{\LaTeX kommt leider mit Code in Macros nicht zurecht. Daher werden code-Schnipsel in der Präambel definiert und in eigene Dateien geschrieben und dann mit dem Befehl \texttt{\textbackslash readcode\{name\}} eingelesen.}{\vspace{-4ex}
%	\readcode{test}
%}
%
%\aufgabe{Kleinere Aufgaben werden am besten mithilfe von tasks gesetzt.}{
%	Beispielsweise zweispaltig:
%	\begin{tasks}(2)
%		\task Ein Aufgabenteil
%		\task Noch ein Aufgabenteil
%		\task ...
%		\task ...
%	\end{tasks}
%	
%	\auftrag[\faCheckSquareO]{Es können auch Symbole mitten im Text eingefügt werden.}
%	Oder auch mehrspaltig:
%	\begin{tasks}(3)
%		\task Ein Aufgabenteil
%		\task Noch ein Aufgabenteil
%		\task ...
%		\task ...
%	\end{tasks}
%}
%
%
%\loesung[symb=\faInfoCircle]{Die Aufgaben und Lösungen können durch fontawesome-Symbole hervorgehoben werden.}{
%	{\LARGE
%	\faEdit\ \faLink\ \faExternalLink\ \faLightbulbO\ \faBook\ \faPencil\ \faGroup\ \faCalendar\ \faCamera\ \faCheck\ \faCheckSquareO\ \faChild\ \faClipboard\ \faClockO\ \faCode\ \faCoffee\ \faComments\ \faCut\ \faDesktop\ \faEnvelope\ \faExclamation\ \faFemale\ \faMale\ \faGift\ \faGraduationCap\ \faHeadphones\ \faHeart\ \faInfo\ \faInfoCircle\ \faLaptop\ \faLeaf\ \faLeanpub\ \faLock\ \faMagic\ \faMapO\ \faMobile\ \faMusic\ \faSend\ \faPaintBrush\ \faPaw\ \faPhone\ \faPlug\ \faPuzzlePiece\ \faPowerOff\ \faPrint\ \faQrcode\ \faQuestion\ \faQuestionCircle\ \faQuestionCircleO\  \faRandom\ \faRocket\ \faSave\ \faStar\ \faStarO\ \faStarHalfO\ \faThumbsOUp\ \faTimes\ \faTrash\ \faTv\ \faVideoCamera\ \faYoutubePlay}
%	
%	vgl. \url{https://packages.oth-regensburg.de/ctan/fonts/fontawesome/doc/fontawesome.pdf}
%}
%
%\abschnitt[\faCode]{Überschriften für wichtige Abschnitte}
%\aufgabe*[pkt=2]{Aufgaben ohne Nummer}{bei Bedarf können auch Punkte vergeben werden.}
%
%\aufgabe[symb=\faYoutubePlay, qr=lehr-lern-labor.info]{Für Links zu Videos o.ä können QR-Codes verwendet werden.}{
%	\blindtext
%}
%
%
%
%\aufgabe[symb=none]{Evtl. werden folgende Listen benötigt:}{%
%\begin{spalten}
%	Single- oder Multiple-Choice
%	\begin{multiplechoice}
%		* Auswahl 1
%		* 2. Wahl
%		*[\faCircle] richtig
%	\end{multiplechoice}
%	Checklist
%	\begin{checklist}
%		* zu erledigen
%		* zu erledigen
%		*[\faCheckSquareO] erledigt
%	\end{checklist}
%	\end{spalten}
%}
\end{document} 
