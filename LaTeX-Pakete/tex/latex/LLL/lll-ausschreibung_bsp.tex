\documentclass[layout=lll]{lll-layout}
% Layouts: lll, schule, scdigi 

\usepackage{lll-ausschreibung}
%\usepackage[print]{lll-ausschreibung}	% E-Mail-Adressen werden im Text angegeben 

%\druckbereich
% ----------------------------------------------------------------------- Infos
\titel{Titel der Ausschreibung}
%\datum{\today}
%\organisation{Alternative Randnotiz}

% ----- Definition von Personen
\def\pA{\person{Person 1}{E-Mail 1}{https://url1}}
\def\pB{\person{Person 2}{E-Mail 2}{https://url2}}

\betreuer{%
	\pA und \pB
}


% ----------------------------------------------------------------------- Inhalt
\begin{document}

% ----- Beispiel Hiwistelle
\hiwi{nach Absprache}

\abschnitt{Hintergrund}
Hier kann (muss aber nicht) eine kleine Einführung in die Problematik stehen.

\abschnitt{Aufgaben}
Beschreibung der Aufgaben

\abschnitt{Voraussetzungen}
Voraussetzungen in Stichworten
\begin{itemize}
	\item Interesse an der Arbeit mit Schüler:innen
	\item Grundlegende Kenntnisse der Fachdidaktik Informatik, beispielsweise aus der Vorlesung \textit{Fachdidaktik Informatik 1}
\end{itemize}

\kontakt

% ----- Beispiel Abschlussarbeit
\newpage
\abschlussarbeit{B.Ed. / M.Ed.}

\abschnitt{Hintergrund}
Hier kann (muss aber nicht) eine kleine Einführung in die Problematik stehen.

\abschnitt{Ziel der Arbeit}
Beschreibung der eigentlichen Aufgabe

\abschnitt{Voraussetzungen}
Voraussetzungen in Stichworten
\begin{itemize}
	\item Interesse an der Arbeit mit Schüler:innen
	\item Grundlegende Kenntnisse der Fachdidaktik Informatik, beispielsweise aus der Vorlesung \textit{Fachdidaktik Informatik 1}
\end{itemize}

\LLL

\kontakt[\pA]

\end{document} 
