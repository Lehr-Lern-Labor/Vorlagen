\documentclass[layout=lll,lsgseite]{lll-layout}

% Layouts: lll, schule (Achtung: unterschiedliche Textbreite)
% Optionen: printlsg (Lösungen nach Aufgabe), lsgseite (Lösungsseite am Ende)
% Papiergröße/Orientierung: z.B. a5paper (a7 - a0), landscape

\usepackage{lll-arbeitsblatt}

\usepackage{listings}
\lstset{	language = html, morecomment	= [s]{<!--}{-->}, alsodigit = {.:;}}


\usepackage{blindtext}

\newif\ifoverleaf
% \overleaftrue	% Overleaf unterstütz filecontents nicht, daher muss der LaTeX-Code entsprechend angepasst werden.
% ----------------------------------------------------------------------- Code
% Codeschnipsel definieren und dann mit \readcode{test} verwenden.
\ifoverleaf
	\renewcommand{\readcode}[1]{\texttt{Code kann auf Overleaf nicht verarbeitet werden.}}
\else
	\begin{filecontents}{code/test.code}
	\begin{lstlisting}
	<html>
		<h1>Hallo Welt</h1>
		<!-- Ein Kommentar -->
		foobar("String")
	</html>
	\end{lstlisting}

	\end{filecontents}
\fi
% ----------------------------------------------------------------------- Infos
\titel{Beispieldokument}
\autor{\href{http://lehr-lern-labor.info}{Lehr-Lern-Labor Informatik}}
\organisation{Beispiele - \lizenz}
\anmerkung{Eine Anmerkung}

%\druckbereich
% ----------------------------------------------------------------------- Inhalt
\begin{document}
\kasten[symb=\faInfo]{Merkkästen für wichtige Infos}{\blindtext}


\kombi[symb=\faLightbulbO]{Hier findet man eine Übersicht über die verfügbaren Makros, die auch direkt Lösungen enthalten.}{
	Lückentext:
	
	Ich bin eine \luecke{Lücke}.	
	
	Kariert:
	
	\kariert[2]{Eine Rechnung mit einer optionalen Kästchen-Anzahl (Höhe)}
	
	Liniert:
	
	\liniert[2]{eine bestimmte Anzahl leerer Linien für Freitext-Antworten}
	
	Blanko:
	
	\blanko[2]{freier Platz für individuelle Antworten (Angbabe in Zeilen)}
}
\kombi{auch mehrspaltiger Satz ist möglich:}{
	\begin{spalten}
		\kariert[3]{linke Spalte}
		\liniert[2]{rechte Spalte}
	\end{spalten}
	\begin{spalten}[3]
		\kariert[3]{linke Spalte}
		\blanko[3]{mittlere Spalte}
		\liniert[2]{rechte Spalte}
	\end{spalten}
}

\newpage
\kombi{\LaTeX kommt leider mit Code in Macros nicht zurecht. Daher werden code-Schnipsel in der Präambel definiert, dabei in eigene Dateien geschrieben und dann mit dem Befehl \texttt{\textbackslash readcode\{name\}} eingelesen.}{\vspace{-4ex}
	\readcode{test}
}

\aufgabe{Kleinere Aufgaben werden am besten mithilfe von tasks gesetzt.}{
	Beispielsweise zweispaltig:
	\begin{tasks}(2)
		\task Ein Aufgabenteil
		\task Noch ein Aufgabenteil
		\task ...
		\task ...
	\end{tasks}
	
	\auftrag[\faCheckSquareO]{Es können auch Symbole mitten im Text eingefügt werden.}
	Oder auch mehrspaltig:
	\begin{tasks}(3)
		\task Ein Aufgabenteil
		\task Noch ein Aufgabenteil
		\task ...
		\task ...
	\end{tasks}
}


\loesung[symb=\faInfoCircle]{Die Aufgaben und Lösungen können durch fontawesome-Symbole hervorgehoben werden.}{
	{\LARGE
	\faEdit\ \faLink\ \faExternalLink\ \faLightbulbO\ \faBook\ \faPencil\ \faGroup\ \faCalendar\ \faCamera\ \faCheck\ \faCheckSquareO\ \faChild\ \faClipboard\ \faClockO\ \faCode\ \faCoffee\ \faComments\ \faCut\ \faDesktop\ \faEnvelope\ \faExclamation\ \faFemale\ \faMale\ \faGift\ \faGraduationCap\ \faHeadphones\ \faHeart\ \faInfo\ \faInfoCircle\ \faLaptop\ \faLeaf\ \faLeanpub\ \faLock\ \faMagic\ \faMapO\ \faMobile\ \faMusic\ \faSend\ \faPaintBrush\ \faPaw\ \faPhone\ \faPlug\ \faPuzzlePiece\ \faPowerOff\ \faPrint\ \faQrcode\ \faQuestion\ \faQuestionCircle\ \faQuestionCircleO\  \faRandom\ \faRocket\ \faSave\ \faStar\ \faStarO\ \faStarHalfO\ \faThumbsOUp\ \faTimes\ \faTrash\ \faTv\ \faVideoCamera\ \faYoutubePlay}
	
	vgl. \url{https://packages.oth-regensburg.de/ctan/fonts/fontawesome/doc/fontawesome.pdf}
}

\abschnitt[\faCode]{Überschriften für wichtige Abschnitte}
\aufgabe*[pkt=2]{Aufgaben ohne Nummer}{bei Bedarf können auch Punkte vergeben werden.}

\aufgabe[symb=\faYoutubePlay, qr=lehr-lern-labor.info]{Für Links zu Videos o.ä können QR-Codes verwendet werden.}{
	\blindtext
}



\aufgabe[symb=none]{Evtl. werden folgende Listen benötigt:}{%
\vspace*{-5ex}
\begin{spalten}
	Single- oder Multiple-Choice
	\begin{multiplechoice}
		* Auswahl 1
		* 2. Wahl
		*[\faCircle] richtig
	\end{multiplechoice}
	Checklist
	\begin{checklist}
		* zu erledigen
		* zu erledigen
		*[\faCheckSquareO] erledigt
	\end{checklist}
	\end{spalten}
}
\end{document} 
