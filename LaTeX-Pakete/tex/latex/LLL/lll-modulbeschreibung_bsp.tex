\documentclass[layout=lll]{lll-layout}
% Optionen: printlsg (Lösungen nach Aufgabe), lsgseite (Lösungsseite am Ende)
\usepackage{lll-modulbeschreibung}

\usepackage{blindtext}
% ----------------------------------------------------------------------- Infos
\titel{Workshopname}
\thema{Thema}
\klasse{ab Klasse X}
\dauer{x Stunden}
\autor{Autor(en)}
\datum{\today}
\format{Präsenz-Workshop}
\vorwissen{benötigtes Vorwissen}
%\vorwissenlang{benötigtes Vorwissen, alternativ falls mehr Platz benötigt wird}
\bp{Bezug zum Bildungsplan, falls vorhanden}


%\druckbereich
% ----------------------------------------------------------------------- Inhalt
\begin{document}
\info

\beschreibung
Eine kurze Beschreibung des gesamten Workshops.

\material{%
	* Stift
	* Papier
	*[\faPlus] je SuS ein Aufgabenblatt
}{%
	* Laptop mit Beamer
	* \texttt{Workshops\_Erklärung.ppt}
}

\ablauf{%
	\phasenwechsel{Teil 1}
	1	& 0:00
	&	\phase{Begrüßung und Einführung}
		\aussage{Formulierungsvorschlag}\n
		Stichwortartige Beschreibung des Ablaufs
	& 	PPT mit Beamer\n\folie{Titelfolie}
	\\\hline
	2	& 0:10
	& 	\phase{Nächste Phase}
	& 	Tafel\n 
	\\\hline
	\phasenwechsel{Optionale Ergänzung (ca. 30 min)}
	5a	& (1:00)
	&	\phase{Vertiefung 1}
	& 	
	\\\hline
	5b	& (1:10)
	&	\phase{Ergebnissicherung 1}
	&	
	\\\hline
	5c	& (1:15)
	&	\phase{Vertiefung 2}
	&
}

\erlaeuterung{%
	1 & \sus kommen und setzen sich
	\\\hline
	2 & Erläuterungen zu den einzelnen Phasen
}

\newpage
\kommentar{%
	1	& Didaktischer Kommentar zu den einzelnen Phasen.
	\\\hline
	2	& ...
}
\end{document} 
